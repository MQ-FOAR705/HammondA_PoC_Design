%--Packages
\documentclass{article}
\usepackage[utf8]{inputenc}
\usepackage{fancyhdr} %--for fancy header/ footers
\usepackage{hyperref}
\usepackage{xcolor}
\usepackage{verbatimbox}
\usepackage{geometry}
\usepackage[pagestyles]{titlesec}



%--Parameters
\title{Proof of Concept I}
\author{Aaron Hammond\\43691455}
\date{September 2019}
\geometry{margin=2.5cm}
\fancyhfoffset{0em}
\definecolor{collink}{RGB}{45, 95, 140}
\titleformat{\section}{\normalfont\Large\bfseries}{}{0em}{}
\hypersetup{
colorlinks,
linkcolor=collink,
linktoc=all
}

%--Page Setup

\pagestyle{fancy}
\fancyhf{}

\rhead{Proof of Concept Design}
\lhead{Digital Humanities}

\fancyfoot[R,RO]{Aaron Hammond | 43691455}
\fancyfoot[L,LO]{Page: \thepage}

\renewcommand{\headrulewidth}{2pt}
\renewcommand{\footrulewidth}{1pt}
%--Commands

\newcommand\HRule{\rule{\textwidth}{1pt}} %-- horizontal lines for formatting

%--
\begin{document}

\begin{center}
%\HRule\\[0.4cm]
\huge{Proof of Concept Design}\\[0.4cm]
\huge{Aaron Hammond}\\[0.3cm]
\large{\today}\\[0.4cm]

\HRule \\[1cm]
\end{center}

%--
\section{Aim}
This document's purpose is to deconstruct the design and process of my proposed solution. This analysis will help to unpack the steps and experiences involved in the solution thus far.

\section{Stories and Acceptance Criteria}
\subsection{User Stories}
\begin{enumerate}
    
    \item As a researcher, I want the ability to convert recorded conversations into a text format.
    
    \item As a researcher, I want the ability to analyse various sources of data together.
    
    \item As a researcher, I want to organise my different sources so their origin can be easily identified.
    
    \item As a researcher, I want the ability to submit all my sources for analysis with a single step or upload.
    
    \item As a researcher, I want to be able to submit multiple files for conversion or upload in order to mitigate repeating tasks.
    
    \item As a researcher, I want the ability to analyse the lexicon of my source.
    
    \item As a researcher, I want to be able to identify and trace the analyis' results.
    
    \item As a researcher, I want the ability to produce a report of the analysis that contains the necessary information.
    
\end{enumerate}

\section{Categorise user stories into themes and identify prerequisites}
The whole solution works according to three main steps. Conversion, Organisation and Analysis. This process depends on the file sources as already having been collected and available locally on my computer. Additionally the following steps must be performed in order from top to bottom. Items written in bold are steps that must be fulfilled to meet the requirements, items in italics are optional.
\subsection{Conversion}
This initial stage is concerned with converting the various types of sources into a single, workable format.

\subsubsection{Interviews}

As a researcher i should be able to:
\begin{enumerate}
    \item Upload file for transcription
    \item Transcribe recordings
    \item\textbf{Retrieve transcribed files as a simple-text file}
    \item\textbf{Save text file into the audio folder}
\end{enumerate}
\subsubsection{Literature}
As a researcher i should be able to:
\begin{enumerate}
    \item Select files for analysis
    \item Load files into a program
    \item\textbf{Convert files into a simple-text files}
    \item\textbf{Export or Save files into the literature folder}
\end{enumerate}

\subsubsection{Fieldnotes}
As a researcher i should be able to:
\begin{enumerate}
    \item Open the Joplin program
    \item Select and export field notes
    \item\textbf{Export to the fieldnotes folder}
\end{enumerate}

\subsection{Organisation}
As a researcher i should be able to:
\begin{enumerate}
    \item Navigate to the parent folder
    \item\textbf{Run a script}
    \item\textbf{See all my sources located within a single folder with their appropriate prefix attached}
\end{enumerate}

\subsection{Analysis}
As a researcher i should be able to:
\begin{enumerate}
    \item Open my preferred software for analysis
    \item\textbf{Import all converted sources for comparison analysis}
    \item\textit{View the results of the analysis and trace their origin}
    \item\textbf{Identify new links between sources}
    \item\textit{Export the results as a report for record keeping}
\end{enumerate}

\section{Conclusion}
The workflow is manageable but I believe there are many avenues for automation. The process can likely be streamlined using commandline tools such as pdftotext and potentially Google's speech to text functionality and potentially a commandline engagement with Joplin. If these avenues prove fruitful, many aspects of the conversion process can be reduced to a single script that a) submits the files to their respective software for conversion b) exports the converted files to the necessary folder c) renames the files with a prefix d) collates all the files together in a single folder for subsequent upload and analysis.\par

\end{document}
